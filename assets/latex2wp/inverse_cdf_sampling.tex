\documentclass[12pt]{article}
\usepackage[pdftex,pagebackref,colorlinks=true,pdfpagemode=none,urlcolor=blue,linkcolor=blue,citecolor=blue,pdfstartview=FitH]{hyperref}

\usepackage{amsmath,amsfonts}
\usepackage{graphicx}
\usepackage{color}
\usepackage{hyperref}
\usepackage{minted}

\setlength{\oddsidemargin}{0pt}
\setlength{\evensidemargin}{0pt}
\setlength{\textwidth}{6.0in}
\setlength{\topmargin}{0in}
\setlength{\textheight}{8.5in}

\setlength{\parindent}{0in}
\setlength{\parskip}{5px}

\input{macrosblog}

\title{Inverse CDF Sampling}
\author{Troy Stribling}

\begin{document}

\iftex
\maketitle
\fi

\iftex
\section{Introduction}
\fi

Inverse \href{https://en.wikipedia.org/wiki/Cumulative_distribution_function}{CDF} sampling is a method for obtaining samples from both discrete and continuous probability distributions
that requires the CDF to be invertable.
The method proposes a CDF value from a Uniform random variable on [0, 1] that is then used as input
into the inverted CDF to generate a sample
with the desired discrete or continuous distribution. Here examples for both cases are discussed.
For the continuous case a proof is given that demonstrates the samples produced have the expected distribution.

\section{Sampling Discrete Distributions}

A discrete probability distribution consisting of a finite set of $N$ probability values is defined by,

\begin{equation}
\label{eq:discrete_distribution}
\{p_1, p_2,\ldots,p_N\}
\end{equation}

with $\sum_{i=1}^N{p_i} = 1.$

The CDF specifies the probability that $i \leq n$ and is given by,
\begin{equation}
\label{eq:discrete_cdf}
P(n)=\sum_{i=1}^n{p_i},
\end{equation}
where $P(N)=1.$

For a given CDF propsal, $P^*$, equation (\ref{eq:discrete_cdf}) can always be inverted by evaluating it for each $n$ and
searching for the value of $n$ that satisfies, $P(n) \geq P^*.$

Consider the distribution,

\begin{equation} \label{eq:discrete}
\left \{\frac{1}{12}, \frac{1}{12}, \frac{1}{6}, \frac{1}{6}, \frac{1}{12}, \frac{5}{12} \right\}
\end{equation}

A Python implementation of a sampler using the Inverse CDF method can be implemented in a few lines of code,

\ifblog
<pre class="EnlighterJSRAW" data-enlighter-language="python" data-enlighter-linenumbers="false">
import numpy

nsamples = 10000
df = numpy.array([1/12, 1/12, 1/6, 1/6, 1/12, 5/12])
cdf = numpy.cumsum(df)

cdf_proposals = numpy.random.rand(nsamples)
samples = [numpy.flatnonzero(cdf >= cdf_proposals[i])[0] for i in range(nsamples)]
</pre>
\fi

\iftex
\begin{minted}[mathescape, frame=lines, framesep=2mm, fontsize=\footnotesize]{python}
import numpy

nsamples = 100000
cdf_proposals = numpy.random.rand(nsamples)
samples = [numpy.flatnonzero(cdf >= cdf_proposals[i])[0] for i in range(nsamples)]
\end{minted}
\fi




\end{document}
